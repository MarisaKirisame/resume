%% start of file `template.tex'.
%% Copyright 2006-2015 Xavier Danaux (xdanaux@gmail.com).
%
% This work may be distributed and/or modified under the
% conditions of the LaTeX Project Public License version 1.3c,
% available at http://www.latex-project.org/lppl/.


\documentclass[10pt,a4paper,sans]{moderncv}        % possible options include font size ('10pt', '11pt' and '12pt'), paper size ('a4paper', 'letterpaper', 'a5paper', 'legalpaper', 'executivepaper' and 'landscape') and font family ('sans' and 'roman')

% moderncv themes
\moderncvstyle{classic}                             % style options are 'casual' (default), 'classic', 'banking', 'oldstyle' and 'fancy'
\moderncvcolor{blue}                               % color options 'black', 'blue' (default), 'burgundy', 'green', 'grey', 'orange', 'purple' and 'red'
%\renewcommand{\familydefault}{\sfdefault}         % to set the default font; use '\sfdefault' for the default sans serif font, '\rmdefault' for the default roman one, or any tex font name
\nopagenumbers{}                                  % uncomment to suppress automatic page numbering for CVs longer than one page

% character encoding
%\usepackage[utf8]{inputenc}                       % if you are not using xelatex ou lualatex, replace by the encoding you are using
%\usepackage{CJKutf8}                              % if you need to use CJK to typeset your resume in Chinese, Japanese or Korean

% adjust the page margins
\usepackage[scale=0.75]{geometry}
%\setlength{\hintscolumnwidth}{3cm}                % if you want to change the width of the column with the dates
%\setlength{\makecvheadnamewidth}{10cm}            % for the 'classic' style, if you want to force the width allocated to your name and avoid line breaks. be careful though, the length is normally calculated to avoid any overlap with your personal info; use this at your own typographical risks...

% personal data
\name{Marisa}{Kirisame}
\email{marisa@cs.utah.edu}                               % optional, remove / comment the line if not wanted
\homepage{www.marisa.moe}                         % optional, remove / comment the line if not wanted
\social[github]{https://github.com/MarisaKirisame/}                              % optional, remove / comment the line if not wanted

% bibliography adjustements (only useful if you make citations in your resume, or print a list of publications using BibTeX)
%   to show numerical labels in the bibliography (default is to show no labels)
%\makeatletter\renewcommand*{\bibliographyitemlabel}{\@biblabel{\arabic{enumiv}}}\makeatother
\renewcommand*{\bibliographyitemlabel}{[\arabic{enumiv}]}
%   to redefine the bibliography heading string ("Publications")
%\renewcommand{\refname}{Articles}

% bibliography with mutiple entries
%\usepackage{multibib}
%\newcites{book,misc}{{Books},{Others}}
%----------------------------------------------------------------------------------
%            content
%----------------------------------------------------------------------------------
\begin{document}
%\begin{CJK*}{UTF8}{gbsn}                          % to typeset your resume in Chinese using CJK
%-----       resume       ---------------------------------------------------------
\makecvtitle

\vspace{-.5in}
\section{Research}
\cvitem{DTR}{Developed an algorithm for gradient checkpointing for large machine learning model. Currently upstreaming to Pytorch. Adopted by Megengine, DELTA, and used in production.}
\cvitem{MemBalancer}{Worked at controlling the garbage collector for V8, the Javascript engine behind Chrome. Utilize concurrent programming and garbage collection knowledge.}
\cvitem{TVM}{Top 20 contributor to high performance ML compiler-runtime. Contributed to the design of Relay, a higher order, differentiable IR. Implemented Algebraic Data Types, Automatic Differentiation, Reference, Pretty Printing, Ahead-Of-Time Compiler, Partial Evaluator, contributed to Type Inference.}

\section{Education}
\cventry{2020--}{PhD in CS}{University of Utah}{Salt Lake City}{}{} % arguments 3 to 6 can be left empty
\cventry{2019--2020}{Master in CS}{University of Washington}{Seattle}{}{} % arguments 3 to 6 can be left empty
\cventry{2015--2019}{Bachelor in CS}{University of Washington}{Seattle}{}{} % arguments 3 to 6 can be left empty

% Publications from a BibTeX file without multibib
%  for numerical labels: \renewcommand{\bibliographyitemlabel}{\@biblabel{\arabic{enumiv}}}% CONSIDER MERGING WITH PREAMBLE PART
%  to redefine the heading string ("Publications"): \renewcommand{\refname}{Articles}

\nocite{*}
\bibliographystyle{unsrt}
\bibliography{publications}                        % 'publications' is the name of a BibTeX file

\section{Projects}
\cvitem{7Tree}{Using CEGIS and Ltac's logical programming capability, build a push-button program synthesizer and verifier for a domain specific problem in Coq.}
\cvitem{Happy-Tree}{A polytypic decision tree in Haskell that work on any True-Sums-Of-Products.}
\cvitem{Ordinary}{A small web game to teach programming. Used Functional Reactive Programming, Nix, Zipper, and GHCJS.}
\cvitem{PE}{Simply Typed Lambda Calculus with reference/product/sum with Bidirectional Type Checking, Partial Evaluation, Automatic Differentiation. Written in MetaOCaml so it can be compiled to OCaml.}
\cvitem{Prover}{An automated theorem prover for first order logic that use Gentzen's Sequent Calculus. Logic Formula represented as Generalized Algebraic Data Type using TMP in C++.}
\cvitem{AI}{Implemented multiple search algorithms in AI Modern Approach, Including A Star, Bidirectional Breath First Search, Constraint Satisfication Programming with K Arch Consistency optimization. Heavily used Iterator Style and Boost to increase efficiency.}
\cvitem{Language}{Fluent in Mandarin, Cantonese, and English.}
\section{Coursework}
\cvlistdoubleitem{Programming Languages, Deep Learning}{Operating Systems}
\cvlistdoubleitem{Advanced Computer Architecture}{Database}
\cvlistdoubleitem{Graduate Theoretical Computer Science}{Systems for Machine Learning}

% Publications from a BibTeX file using the multibib package
%\section{Publications}
%\nocitebook{book1,book2}
%\bibliographystylebook{plain}
%\bibliographybook{publications}                   % 'publications' is the name of a BibTeX file
%\nocitemisc{misc1,misc2,misc3}
%\bibliographystylemisc{plain}
%\bibliographymisc{publications}                   % 'publications' is the name of a BibTeX file

\clearpage
\end{document}


%% end of file `template.tex'.
